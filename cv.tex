\documentclass[margin,line]{res}
\usepackage{hyperref}
%\usepackage{CJK}
\usepackage[fontset=ubuntu]{ctex}
\usepackage{marvosym} % For cool symbols.
%\usepackage[hyperref,UTF8]{ctex}
\pagestyle{plain} \pagenumbering{Roman}
\usepackage{enumitem}

\usepackage{fontspec,xunicode,xltxtra}
\usepackage{xeCJK} 
\setCJKmainfont[BoldFont={},ItalicFont={SIMKAI.TTF}]
{SIMSUN.TTC}
%\setCJKsansfont{SimHei}
%\setCJKmonofont{[FangSong]}

\setCJKfamilyfont{zhsong}{SIMSUN.TTC}
\setCJKfamilyfont{zhhei}{SimHei}
\setCJKfamilyfont{zhkai}{[simkai.ttf]}
\setCJKfamilyfont{zhfs}{[FangSong]}
% \setCJKfamilyfont{zhli}{LiSu}
% \setCJKfamilyfont{zhyou}{YouYuan}

%\newcommand*{\songti}{\CJKfamily{zhsong}} % 宋体
%\newcommand*{\heiti}{\CJKfamily{zhhei}}   % 黑体
%\newcommand*{\kaishu}{\CJKfamily{zhkai}}  % 楷书
%\newcommand*{\fangsong}{\CJKfamily{zhfs}} % 仿宋
\setmainfont{Times New Roman}
%\setCJKmainfont{SimSun} % 设置缺省中文字体
%\setCJKmonofont{SimHei} % 设置等宽字体
%\oddsidemargin -.5in
%\evensidemargin -.5in
%\textwidth=6.0in
%\itemsep=0in
%\parsep=0in

\oddsidemargin 0.1in \evensidemargin 0in \textwidth=4.8in
\textheight=8.6in
\itemsep=0in
\parsep=0in



\newenvironment{list1}{
  \begin{list}{\ding{113}}{%
      \setlength{\itemsep}{0in}
      \setlength{\parsep}{0in} \setlength{\parskip}{0in}
      \setlength{\topsep}{0in} \setlength{\partopsep}{0in}
      \setlength{\leftmargin}{0.17in}}}{\end{list}}
\newenvironment{list2}{
  \begin{list}{$\bullet$}{%
      \setlength{\itemsep}{0in}
      \setlength{\parsep}{0in} \setlength{\parskip}{0in}
      \setlength{\topsep}{0in} \setlength{\partopsep}{0in}
      \setlength{\leftmargin}{0.2in}}}{\end{list}}


\begin{document}
%\begin{CJK*}{GBK}{song}
\name{\LARGE{SU Zhou (苏舟)} \vspace*{.20in}}
%\textbf{•}\name{苏舟 \vspace*{.20in}}
\begin{resume}


	\section{\sc 个人信息}
	\vspace{.05in}
	\begin{tabular}{@{}p{2.6in}p{3in}}
		手机: (+86) 159-0127-9099 \
		 & E-mail:\href{mailto:suhmily@gmail.com}{suhmily\textrm{@}gmail.com} \\
	\end{tabular}



	\section{\sc 工作经历}
	\textbf{快手社区科学部},算法工程师 \hfill{2021.01 - 至今}\\
	% \begin{list2}
	% \item[] 负责同城页直播营收业务算法建设
	% \item[] 负责内容增长业务及算法建设 
	% \end{list2}
	\textbf{腾讯微信推荐产品中心},应用研究员 \hfill{2017.12 - 2021.01} \\
	% \begin{list2}
	% \item[] 负责研发微信看一看中视频推荐以及视频理解中台建设
	% \item[] 负责研发微信搜一搜中基础相关性算法
	% \end{list2}
	\textbf{英特尔研究院},研究员 \hfill{2015.10 - 2017.12}\\
	% \begin{list2}
	% \item[] 负责研发英特尔平台上的深度学习相关算法 
	% \end{list2}

	\section{\sc 教育经历}
	\textbf{卡耐基梅隆大学}, 信息科学,硕士 \hfill{2014 - 2015} \\
	\textbf{清华大学}, 计算机科学与技术, 学士 \hfill{2010 - 2014}


	\section{\sc 项目经历}
	\textbf{快意大语言模型预训练} \hfill{2023}
	\begin{description}
		% \item \textbf{责任描述}: 作为预训练主要负责人之一,优化快意大语言模型的模型效果
		\item \textbf{项目描述}:
		      \begin{list2}
			      \item code效果优化:设计实现 code 数据的处理、清洗、组织、采样的方式,提升模型的代码能力。
				  	1) 代码数据清洗:使用代码规范化、代码注释清洗、代码语言检测等方式提升代码质量, 最终筛选 89 种常用代码语言。
					2) 代码数据组织:使用语法树分析、 Text-Code 互译、代码注释清洗和补充等方式提升代码组织能力。
					3) 代码数据采样:使用代码数据增强、 按照星标分层采样、使用质量模型代码数据筛选等方式提升代码数据质量。
			      \item 模型小型化优化:使用模型蒸馏、COT 等方式提升 1.3b 规模模型的推理效果。
				  	1) 模型蒸馏:使用模型蒸馏技术,将大模型的置信度信息引入小模型的训练,提升小模型的逻辑能力。
					2) COT:针对小模型复杂推理能力的弱项,使用 COT 技术,补充复杂知识的过程和逻辑链条,提升模型推理能力。
			      \item 构造数据:
				  	1) 数据生成:使用 llm 生成高质量训练数据,包含垂类课程数据、推理过程数据、稀缺知识数据等,提升模型效果,缓解高质量数据的稀缺性问题。
					2) 数据挖掘:区分事实性知识和教育性知识,使用llm 质量模型挖掘有教育价值的数据,提升模型的通用能力。
		      \end{list2}
	\end{description}

	\textbf{快意多模态模型} \hfill{2022}
	\begin{description}
		% \item \textbf{责任描述}:作为SFT 方向负责人,从 0到 1 搭建快意多模态大模型 SFT框架
		\item \textbf{项目描述}:
		      \begin{list2}
			      \item 数据体系搭建:构建SFT任务体系,包含5 大类、37 项任务,建设完整的数据链路,以及搭建数据可视化平台。
			      \item 模型结构优化:使用 deepspeed 分布式训练框架,基于 llava 的多模态模型结构,针对快意多模态模型的特点,设计模型结构,提升模型效果。
		      \end{list2}
	\end{description}

	\textbf{基于强化学习的冷启动推荐} \hfill{2021}
	\begin{description}
		% \item \textbf{责任描述}: 作为项目负责人,从0到1构建冷启动算法业务。
		\item \textbf{项目描述}:
		      \begin{list2}
			      \item 基于强化学习算法,设计搭建内容冷启动视频撬动算法,基于视频的分发速度提升视频未来曝光的长期收益,提升视频成长价值。
			      \item 将内容冷启动算法应用到推荐系统的内容挖掘、召回、排序等各个模块, 提升视频的冷启动效率,拓宽推荐视频候选的入水口。
		      \end{list2}
	\end{description}

	\textbf{Visual knowledge memory netowrk for visual question answering} \hfill{2017}
	\begin{description}
		%\item 关键技术:CNN(Resnet-152), LSTM, Knowledge graph, question answering
		%\item 编程语言:Torch, Python 
		%\item 开发环境:Linux
		% \item \textbf{责任描述}:作为主要负责人,设计算法框架, 实现模型代码,参加Visual QA challenge 2017 竞赛, 撰写\emph{CVPR} 2018相关论文
		\item \textbf{项目描述}:本项目提出了一种针对视觉问答的全新的框架 Visual knowledge memory netowrk (VKMN),融合了结构注意力模型 (visual structured attention), 知识库 (knowledge graph) 和键值记忆网络 (key-value memroy network)。 VKMN 模型通过对视觉信息知识库的构建和查找,对视觉问题答案预测进行了信息补充和常识约束,使视觉问答的结果更符合人类的常识,提高答案预测的准确性。本项目参加了由弗吉尼亚理工大学举办的 Visual QA Challenge 2017, 并取得预测准确率的第五名, 相关论文``Learning Visual Knowledge Memory Networks for Visual Question Answering'' 被\emph{CVPR} 2018接收。
	\end{description}


	\section{\sc 论文列表}
	\begin{enumerate}[leftmargin=*]
		\item \textbf{Z. Su},  C. Zhu, Y. Dong, D. Cai, Y. Chen and J. Li ``Learning Visual Knowledge Memory Networks for Visual Question Answering,'' accepted by \emph{CVPR} 2018.

		\item Z. Shen, J. Li, \textbf{Z. Su},  M. Li, Y. Chen, Y. Jiang and X. Xue ``Weakly Supervised Dense Video Captioning,'' accepted by \emph{CVPR} 2017.

		\item P. Cui, Z. Wang and \textbf{Z. Su}, ``What Videos Are Similar With You? Learning a Common Attributed Representation for Video Recommendation,'' accepted by \emph{ACM Multimedia}, 2014.

		\item \textbf{Z. Su},H. Wei and S. Wei, ``Crowd Event Perception Based On Spatio-Temporal Weber Field,'' accepted by \emph{JECE}, 2014.

	\end{enumerate}

	\section{\sc 主要奖项}
	\begin{list2}
		\item \textbf{2023洛子峰奖, 快手} \hfill{2023}
		\\\emph{快手技术项目最高奖,快意大语言模型项目}
		\item \textbf{年度绩效A, 快手} \hfill{2022}
		\\\emph{部门最高绩效,在贡献业务结果与模型能力方面有杰出表现。}
		\item \textbf{优秀员工奖, 微信} \hfill{2018 Q1, 2019 Q2}
		\\\emph{在完成提交出色业务成果方面有杰出表现。}
		\item \textbf{Visual Question Answering竞赛, 弗吉尼亚理工大学} \hfill{2017}
		\\\emph{在图片视觉问答比赛中获得准确率第五名。}
		\item \textbf{MSR-VTT竞赛, 微软} \hfill{2016}
		\\\emph{在视频描述自动生成比赛中获得人工评测第四名。}
		\item \textbf{Deliver Award, 英特尔中国研究中心} \hfill{2016, 2017}
		\\\emph{在带领团队提交出色的业务成果方面有杰出表现。}

	\end{list2}

\end{resume}
%\end{CJK*}
\end{document}
